{\chapter{Code snippets}\label{ch:scope}}
\epigraph{\textit{\normalsize “By far the greatest danger of Artificial Intelligence is that people conclude too early that they understand it.”}}{\textit{ \normalsize Eliezer Yudkowsky,\\ Machine Intelligence Research Institute}}

The internal architectures of all four GANs are similarly designed. Here we take DCGAN to showcase the code.

\par\bigskip 
The main code consists of a class (DCGAN) which contains the following five functions:
\par\bigskip

\begin{enumerate}
    \item Initialization: Calls build\_generator and build\_discriminator and makes a combined model
    \item Build\_Generator: Creates a generator model
    \item Build\_Discriminator: Creates a discriminator model
    \item Train: Takes the input images and starts training, prints training progress with metrics
    \item Save\_Imgs: Saves a grid of generated images at specific epochs
\end{enumerate}

\section{Initialization} % (fold)
\label{sec:initialization}
\begin{lstlisting}[basicstyle=\scriptsize,language=Python]
def __init__(self):
    # Input shape
    self.img_rows = 64
    self.img_cols = 64
    self.channels = 3
    self.img_shape = (self.img_rows, self.img_cols, self.channels)
    self.latent_dim = 100

    optimizer = Adam(0.0002, 0.5)

    # Build and compile the discriminator
    self.discriminator = self.build_discriminator()
    self.discriminator.compile(loss='binary_crossentropy',
        optimizer=optimizer,
        metrics=['accuracy'])

    # Build the generator
    self.generator = self.build_generator()

    # The generator takes noise as input and generates imgs
    z = Input(shape=(100,))
    img = self.generator(z)

    # For the combined model only train the generator
    self.discriminator.trainable = False

    # The discriminator takes generated images as input and determines validity
    valid = self.discriminator(img)

    # The combined model  (stacked generator and discriminator)
    self.combined = Model(z, valid)
    self.combined.compile(loss='binary_crossentropy', optimizer=optimizer)
\end{lstlisting}

% section initialization (end)

\section{Building Generator} % (fold)
\label{sec:building_generator}

\begin{lstlisting}[basicstyle=\scriptsize,language=Python]
def build_generator(self):
    """
    Build generator which takes noise (a tensor of size 100) as input,
    and produces an RGB image of size (64 x 64) .
    """

    # Create a model in which one can add layers sequentially
    model = Sequential()

    # Add a densely connected layer to the model, activation function of ReLu
    model.add(Dense(128 * 8 * 8, activation="relu", 
        input_shape=(self.latent_dim,)))
    model.add(Reshape((8, 8, 128)))

    # DeConv layer one starts
    model.add(BatchNormalization(momentum=0.8))
    model.add(UpSampling2D())
    model.add(Conv2D(128, kernel_size=3, padding="same"))
    model.add(Activation("relu"))

    # DeConv layer two starts
    model.add(BatchNormalization(momentum=0.8))
    model.add(UpSampling2D())
    model.add(Conv2D(64, kernel_size=3, padding="same"))
    model.add(Activation("relu"))

    # DeConv layer three starts
    model.add(BatchNormalization(momentum=0.8))
    model.add(UpSampling2D())
    model.add(Conv2D(32, kernel_size=3, padding="same"))
    model.add(Activation("relu"))

    # Final output layer
    model.add(BatchNormalization(momentum=0.8))
    model.add(Conv2D(self.channels, kernel_size=3, padding="same"))
    model.add(Activation("tanh"))

    # Print model summary
    model.summary()

    # Use functional API to make model
    noise = Input(shape=(self.latent_dim,))
    img = model(noise)

    # Return functional model
    return Model(noise, img)

\end{lstlisting}
% section building_generator (end)

\section{Building Discriminator} % (fold)
\label{sec:building_discriminator}
\begin{lstlisting}[basicstyle=\scriptsize,language=Python]
def build_discriminator(self):
    """
    Discriminator takes real/generated images and outputs its prediction.
    """

    # Define input shape for our network
    img = Input(shape=self.img_shape)
    
    # First ConvLayer outputs a 56x56x256 matrix
    x = Conv2D(filters=256, kernel_size=9, strides=1, 
        padding='valid', name='conv1')(img)
    x = LeakyReLU()(x)
    x = BatchNormalization(momentum=0.8)(x)

    # Capsule architecture starts

    # First layer: PrimaryCaps
    x = Conv2D(filters=8 * 32, kernel_size=9, strides=2, 
        padding='valid', name='primarycap_conv2')(x)

    # Primary capsule has collections of activations which denote orientation
    # while intensity of the vector which denotes the presence of the digit)
    x = Reshape(target_shape=[-1, 8], name='primarycap_reshape')(x)

    # Output a number between 0 and 1 for each capsule 
    # where the length of the input decides the amount
    x = Lambda(squash, name='primarycap_squash')(x)
    x = BatchNormalization(momentum=0.8)(x)

    # Second layer: DigitCaps
    # This is a modified form of the standard CapsNet DigitCaps architecture 
    # where we have replaced the multiple capsules with a single capsule of 
    # densely connected neural network.
    x = Flatten()(x)

    # Dynamic Routing
    # uhat = prediction vector, u * w
    # w = weight matrix but will act as a dense layer
    # u = output from a previous layer
    uhat = Dense(160, kernel_initializer='he_normal', 
        bias_initializer='zeros', name='uhat_digitcaps')(x)

    # softmax will make sure that each weight c_ij is a non-negative number 
    # and their sum equals to one
    c = Activation('softmax', name='softmax_digitcaps1')(uhat) 

    # s_j (output of the current capsule level) = uhat * c
    c = Dense(160)(c) # compute s_j
    x = Multiply()([uhat, c])

    # Squashing the capsule outputs creates severe blurry artifacts, 
    # thus we replace it with Leaky ReLu.
    s_j = LeakyReLU()(x)


    c = Activation('softmax', name='softmax_digitcaps2')(s_j) 
    c = Dense(160)(c)
    x = Multiply()([uhat, c])
    s_j = LeakyReLU()(x)

    c = Activation('softmax', name='softmax_digitcaps3')(s_j) 
    c = Dense(160)(c)
    x = Multiply()([uhat, c])
    s_j = LeakyReLU()(x)

    # Final dense layer output a binary classification
    pred = Dense(1, activation='sigmoid')(s_j)
    return Model(img, pred)
\end{lstlisting}

% section building_discriminator (end)

\section{Train} % (fold)
\label{sec:train}
\begin{lstlisting}[basicstyle=\scriptsize,language=Python]
def train(self, epochs, batch_size=128, save_interval=50):
    
    half_batch = int(batch_size / 2)

    for epoch in range(epochs):
        # ---------------------
        #  Train Discriminator
        # ---------------------
        cnt=0
        train_datagen = ImageDataGenerator(rescale=1./255)
        train_generator = train_datagen.flow_from_directory('data',
            target_size=(64, 64),batch_size=half_batch,class_mode=None)
        for x in train_generator:
            # Sample noise and generate a half batch of new images
            noise = np.random.normal(0, 1, (half_batch, 100))
            gen_imgs = self.generator.predict(noise)

            # Train the discriminator 
            # (real classified as ones and generated as zeros)
            d_loss_real = self.discriminator.train_on_batch
                (x, np.ones((half_batch, 1)))
            d_loss_fake = self.discriminator.train_on_batch
                (gen_imgs, np.zeros((half_batch, 1)))
            d_loss = 0.5 * np.add(d_loss_real, d_loss_fake)

            # ---------------------
            #  Train Generator
            # ---------------------

            # Sample generator input
            noise = np.random.normal(0, 1, (batch_size, 100))

            # Train the generator 
            # (wants discriminator to mistake images as real)
            g_loss = self.combined.train_on_batch
                (noise, np.ones((batch_size, 1)))

            # Plot the progress
            if(cnt%save_interval==0):
              print ("%d [D loss: %f, acc.: %.2f%%] [G loss: %f]" 
                % (epoch, d_loss[0], 100*d_loss[1], g_loss))
            
            # If at save interval => save generated image samples
            if(cnt%save_interval==0):
              self.save_imgs(cnt)
            cnt+=1
\end{lstlisting}
% section train (end)

\section{Save Imgs} % (fold)
\label{sec:save_imgs}
\begin{lstlisting}[basicstyle=\scriptsize,language=Python]
def save_imgs(self, epoch):
    r, c = 5, 5
    noise = np.random.normal(0, 1, (r * c, 100))
    gen_imgs = self.generator.predict(noise)

    # Rescale images 0 - 1
    gen_imgs = 0.5 * gen_imgs + 0.5

    fig, axs = plt.subplots(r, c)
    #fig.suptitle("DCGAN: Generated digits", fontsize=12)
    cnt = 0
    for i in range(r):
        for j in range(c):
            axs[i,j].imshow(gen_imgs[cnt, :,:,0], cmap='gray')
            axs[i,j].axis('off')
            cnt += 1
    fig.savefig("images/images_%d.png" % epoch)
    plt.close()
\end{lstlisting}
% section save_imgs (end)