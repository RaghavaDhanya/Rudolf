\chapter{Introduction}\label{ch:introduction}

\epigraph{\normalsize\textit{ "What I cannot create, I do not understand."}}{\normalsize\textit{ Richard Feynman}}

One of the main aspirations of Artificial Intelligence is to develop algorithms and techniques that enrich computers with ability to understand our world. Generative models are one of the most promising approaches towards achieving this goal.\par\bigskip
A generative model is a mathematical or statistical model to generate all values of a phenomena. To train such a model, we first collect a large amount of data in some domain (e.g., think millions of images, sentences, or sounds, etc.) and then train a model to generate data like it.\par\bigskip
A generative algorithm models how data was generated to classify a signal. It poses the question: according to my generation hypotheses, which category is most likely to generate this signal? A discriminative algorithm does not care about how the data was generated, it just classifies a given signal. A generative model learns the joint probability distribution $p(x,y)$ while a discriminative model learns the conditional probability distribution $p(y|x)$ “probability of y given x”.\par\bigskip
The trick is that the neural networks that we use as generating models have a significantly smaller number of parameters than the amount of data on which we train them, so the models are forced to effectively discover and internalize the essence of the data to generate it.\par\bigskip
There are multiple approaches to build a generative models
\begin{itemize}
  \item Generative adversarial networks (GANs) are a class of generative algorithms used in unsupervised machine learning, implemented by a system of two neural networks competing in a zero-sum game framework. They were presented by Ian Goodfellow \textit{et al}. \cite{gan}. This technique can generate photographs that seem at least superficially authentic to human observers, having many realistic features (though in tests people can tell real from generated in many cases).
  \item Variational Autoencoders (VAEs) allow us to formalize this problem in the framework of probabilistic graphical models where we are maximizing a lower bound on the log likelihood of the data
  \item Autoregressive models such as PixelRNN, on the other hand train a network that models the conditional distribution of every individual pixel given previous pixels (to the left and to the top). This is similar to plugging the pixels of the image into a char-rnn, but the RNNs runs both horizontally and vertically over the image instead of just a 1D sequence of characters.
\end{itemize} \par
Generative Adversarial Networks, which we already discussed above, pose the training process as a game between two distinct networks: a generator network (as seen above) and a second discriminative network that tries to classify samples as either coming from the true distribution $p(x)$ or the model distribution $\hat{p}(x)$. Every time the discriminator notices a difference between the two distributions the generator adjusts its parameters slightly to make it go away, until at the end (in theory) the generator exactly reproduces the true data distribution and the discriminator is guessing at random, unable to find a difference.
