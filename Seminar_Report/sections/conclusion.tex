\chapter{Conclusion}\label{ch:conclusion}
Machine learning and quantum information processing are two large, technical fields that must both be well understood before their intersection is explored. To date, the majority of QML research has come from experts in either one of these two fields and, as such, oversights can be made. The most successful advances in QML have come from collaborations between CML and quantum information experts, and this approach is highly encouraging. These early results highlight the promise of QML, where not only are time and space scalings possible but also more accurate classifiers.\par\bigskip
There are several important issues to overcome before the practicality of QML becomes clear. The main problem is efficiently loading large sets of arbitrary vectors into a quantum computer. The solution may come from a physical realization of QRAM, however as discussed previously, it is currently unclear whether this will be possible.\par\bigskip
Despite the potential limitations, research into the way quantum systems can learn is an interesting and stimulating pursuit. 
Considering what it means for a quantum system to learn could lead to novel quantum algorithms with no classical analogue.
