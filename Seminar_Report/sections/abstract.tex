\chapter*{Abstract\markboth{Abstract}{Abstract}}\label{ch:abstract}
\addcontentsline{toc}{chapter}{Abstract}

Recently, the increase in computational power and available data, as well as algorithmic advances in machine learning techniques, have led to impressive results in regression, classification, data generation, and reinforcement learning. Despite these successes, the proximity of the physical boundaries of chip manufacturing as well as the growing size of data sets is driving an increasing number of researchers to explore the possibility of harnessing the power of quantum computing to accelerate classical machine learning algorithms\par\bigskip
Machine learning tasks often involve problems with manipulation and classification of a large number of vectors in large spaces. Conventional algorithms for solving such problems typically take a polynomial time in the number of vectors and the size of the space. Quantum computers are good for handling high-dimensional vectors in large tensor product spaces\par\bigskip
The field of quantum machine learning explores how to design and implement quantum algorithms that makes machine learning faster. Recent work has produced quantum algorithms that could serve as building blocks for machine learning programs, but the hardware and software challenges are still considerable.