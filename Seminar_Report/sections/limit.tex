{\chapter{Limitations}\label{ch:limitations}}
Let's tackle the question of when to use quantum machine learning. It turns out that the quantum computers can be a real lifesaver for the big data people, you could query all of the data at a single time, not just sample them in batches. This is a very big improvement as most of the heavy data retrieval is done in batches and this adds to significant delay in what has to be achieved. It would also do great for those who are working with graph computing where you could factor in the complexity of many-to-many relationships that would otherwise require endless joins with relational data models. They will soon be known for breaking most encryptions as factorization will be very fast using algorithms like Shor's algorithm.\par\bigskip
On the downside, quantum computers are not universal computers. They cannot replace classic computers. The whole idea of quantum computers is to exploit the quantum physics in performing certain complex computations very fast. For most of the mundane programs and tasks, the classic computer would actually perform better than quantum computers. But quantum computers would substantially improve the runtime in case of certain computations that would have been practically impossible with classic computers.\par\bigskip
Let's extend this further and look into the limitation of QC.
In physics, quantum noise refers to the uncertainty of a physical quantity that is due to its quantum origin. In certain situations, quantum noise appears as shot noise. To explain what shot noise is, consider a simple experiment of tossing a coin and counting the occurrences of heads and tails, the numbers of heads and tails after a great many throws will differ by only a tiny percentage, while after only a few throws outcomes with a significant excess of heads over tails or vice versa are common; if an experiment with a few throws is repeated over and over, the outcomes will fluctuate a lot.\par\bigskip
These random fluctuations may be caused from heat in the qubits. Another reason might be that the quantum-mechanical processes will occasionally flip or randomize the state of a qubit, potentially derailing a calculation. This is a hazard in classical computing too, but it's not hard to deal with-- you just keep two or more backup copies of each bit so that a randomly flipped bit stands out as the odd one out. \par\bigskip
Researchers working on quantum computers have created strategies to deal with the noise. But these strategies impose a huge debt of computational overhead-- all your computing power goes to correcting errors and not to running your algorithms. Current error rates significantly limit the lengths of computations that can be performed. The major issue in quantum error correction is that Superpositions can only be sustained as long as you don't measure the qubit's value. If you make a measurement, the superposition collapses to a definite value: 1 or 0. One ingenious scheme involves looking indirectly, by coupling the qubit to another "ancilla" qubit that doesn't take part in the calculation but that can be probed without collapsing the state of the main qubit itself. It's complicated to implement, though. To create one "logical qubit" around 10,000 of today's physical qubits is needed, which is impractical. Some researchers think that the problem of error correction will prove intractable and will prevent quantum computers from achieving the grand goals predicted for them.\par\bigskip
The known quantum algorithms for machine learning problems suffer from a number of caveats that limit their practical applicability. Classical computers and quantum computers are very different on the lowest level. Classical computers use electrons while quantum computers use photons for the same purpose. Also, all the logical gates used in a classical computer are non-reversible (except the NOT gate), while in case of quantum computers all the gates are reversible. So, classical computers and quantum computers are very different, not just in architecture, but in the way they perform their operations. This is the reason why quantum computers are good in performing calculations but can't run even our 1st and 2nd generation of computer games. \par\bigskip
Suppose you are in the middle of a maze and want to find the exit path. In case of classical computer you will try each and every path one by one until you find the exit. And think about this in the way of algorithms used, like Branch and Bound, Backtracking, etc. Things are different and more surprising with quantum computers. In our little puzzle a quantum computer will try all the paths simultaneously in the first time. And no matter what kind of maze is given it will find the path in the first iteration. And, trying all the paths is due to the fact that quantum computer is in superposition of exponentially many states, not the processor actually computing all the paths simultaneously. It is very important that you understand the difference between the two. Quantum Computers, instead of checking possibilities one by one, create a uniform superposition over all possibilities and repeatedly and destructively interfere states that are not solutions. All quantum algorithms [can] have at-least two parts. One part is often like the classical part and be run on classical computers without any difference in output. The other part is the quantum part and a correct result can only come from quantum computer. 
One of the most well known algorithm is Shor's algorithm for factoring. Shor's algorithms runs exponentially faster than the best known classical algorithm for factoring. Grover's algorithm searches an unstructured database (or an unordered list) with N entries, for a marked entry, using only $O({\sqrt {N}})$ queries instead of the $O(N)$ queries required classically. \par\bigskip
Quantum computers have the potential to do some kinds of calculation with unprecedented speed, as small-scale demonstrations have confirmed. However, to perform most of these calculations effectively these machines will eventually need to access something resembling random access memory (RAM) -- a large store of quantum information that can be selectively accessed. Ordinary RAM contains a large array of memory cells, each holding one bit of information -- a binary 0 or 1. To check the contents of particular cell, a computer accesses it using its address -- a string of bits that identifies the cell's location. A quantum computer uses not bits but qubits, which can be a blend of 0 and 1 -- a quantum superposition of the two states. Therefore, in quantum RAM, the address qubits would not identify a single memory cell but a certain superposition of all possible memory cells.  For large data access, large QRAM might be required which might not be implementable.

