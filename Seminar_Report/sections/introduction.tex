\chapter{Introduction}\label{ch:introduction}

Machine learning is highly difficult. It’s what mathematicians call an “NP-hard” problem. That’s because building a good model is really a creative act. As an analogy, consider what it takes to architect a house. You’re balancing lots of constraints -- budget, usage requirements, space limitations, etc. -- but still trying to create the most beautiful house you can. A creative architect will find a great solution. Mathematically speaking the architect is solving an optimization problem and creativity can be thought of as the ability to come up with a good solution given an objective and constraints. Classical computers aren’t well suited to these types of creative problems.Machine learning algorithms are tasked with extracting meaningful information and making predictions about data. In contrast to other techniques, these algorithms construct and/or update their predictive model based on input data. The applications of the field are broad, ranging from Spam filtering to image recognition, demonstrating a large market and wide societal impact.\par\bigskip
In recent years, there have been a number of advances in the field of quantum information showing that particular quantum algorithms can offer a speedup over their classical counterparts. Execution time is just one concern of learning algorithms. Quantum optimizations are capable of finding the global minimum of a non convex objective functions in a discrete search space. Storage capacity is also of interest. Quantum associative memories store exponentially more patterns than a classical Hopfield networks. In addition to supplying exponential speed-ups in both number of vectors and their dimension, quantum machine learning allows enhanced privacy: In a Quantum clustering algorithm only $O(log(MN))$ calls to the quantum data-base are required to perform cluster assignment, while $O(MN)$ are required to uncover the actual data \cite{kmeans}.