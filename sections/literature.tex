\chapter{Literature Survey}\label{ch:literature_survey}
\epigraph{\textit{\Large “Adversarial training is the coolest thing since sliced bread”}}{\textit{ \large Yann LeCun,\\ Director of AI Research at Facebook and Professor at NYU}}


GANs were first introduced by Ian Goodfellow et al. \cite{gan} in Neural Information Processing Systems 2014. The paper proposes a completely new framework for estimating generative models via an adversarial process. In this process two models are simultaneously trained. According to \cite{gan} the network has a generative model G that captures the data distribution, and a discriminative model D that estimates the probability that a sample came from the training data rather than G. This original work by Ian Goodfellow uses fully connected neural networks in the generator and the discriminator. 
\par\bigskip


Since then, there has been tremendous advancements in Deep Learning. A convolutional neural network (CNN, or ConvNet) \cite{imagenet} is a class of deep, feed-forward artificial neural networks that has successfully been applied to analyzing visual imagery. The convolution layer parameters consist of a set of learnable filters, also called as kernels, which have a small receptive field, but they extend through the full depth of the input volume. As a result, the network learns filters that activate when it detects some specific type of feature at some spatial position in the input.
\par\bigskip

A breakthrough development that occurred in Adversarial Networks was the introduction of “Deep Convolutional Generative Adversarial Networks” by Alec Radford et al, ICLR, 2016 in 2016 in ICLR\cite{dcgan}. He applied a list of empirically validated tricks as the substitution of pooling and fully connected layers with convolutional layers.
\par\bigskip

The power of the features encoded in the latent variables was further explored by Chen at al. \cite{infogan}. They propose an algorithm which is completely unsupervised, unlike previous approaches which involved supervision, and learns interpretable and disentangled representations on challenging datasets. Their approach only adds a negligible computation cost on top of GAN and is easy to train.
\par\bigskip

Today, most GANs are loosely based on the former shown DCGAN \cite{dcgan} architecture. Many papers have focused on improving the setup to enhance stability and performance. Many key insights was given by Salimans et al.\cite{improvedgan}, like Usage of convolution with stride instead of pooling, Usage of Virtual Batch Normalization, Usage of Minibatch Discrimination in DD, Replacement of Stochastic Gradient Descent with Adam Optimizer [6], Usage of one-sided label smoothing.
\par\bigskip

Another huge development came with the introduction of Wasserstein GANs by Martin Arjovsky \cite{wgan} . He introduced a new algorithm named WGAN, an alternative to traditional GAN training. In this new model, he showed that the stability of learning can be improved, remove problems like mode collapse, and provide good learning curves useful for debugging and hyperparameter searches.
\par\bigskip

This recently proposed Wasserstein GAN (WGAN) \cite{wgan} makes progress toward stable training of GANs, but sometimes can still generate only low-quality images or fail to converge. 
Ishaan Gulrajani with Martin Arjovsky proposed an alternative in \cite{improvedwgan} to fix the issues the previous GAN faced. This proposed method performs better than standard WGAN and enables stable training of a wide variety of GAN architectures with almost no hyperparameter tuning, including 101-layer ResNets\cite{deepresidual} and language models over discrete data.
\par\bigskip

A big breakthrough in the field of Deep Learning came with the introduction of CapsNets or Capsule Networks[10] by the Godfather of Deep Learning, Geoffrey Hinton. CNNs perform exceptionally great when they are classifying images which are very close to the data set. If the images have rotation, tilt or any other different orientation then CNNs have poor performance. This problem was solved by adding different variations of the same image during training.
\par\bigskip